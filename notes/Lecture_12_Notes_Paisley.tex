\documentclass[a4paper, 10pt]{article}
\usepackage[margin=0.45in]{geometry}
\usepackage{graphicx,amssymb,amstext,amsmath,amsthm,enumerate,float,textcomp}

\begin{document}

\begin{center}
\huge STA 250. Fall, 2013. \\ 
\Large Lecture 12: Optimization + EM Lecture $\#$ 1.\\ \small Transcribed by Eliot Paisley. 11/6/13
\end{center}
\hrule 
\normalsize 
\vspace{0.5in}

\underline{Introduction:}
\begin{itemize}
	\item We'll spend only a short time on optimization ... this is really an EM module. 
	\item To fit any non-standard statistical models (i.e., outside of just \texttt{lm}, \texttt{glm}, or \texttt{lme}), we need know a little bit about numerical methods. We've already seen one example, the Metropolis-Hastings algorithm. 
	\item For Bayes problems we use Markov-Chain Monte Carlo (MCMC) methods, while for maximum likelihood (ML) problems we need to maximize a non-standard function. This entire module is about maximizing these `difficult' likelihoods (or posteriors). 
\end{itemize}
To begin, we start by looking at some common optimization algorithms; Bisection, Newton-Raphson, and Scoring. \\
\emph{Note: we're actually looking at root-finding algorithms. i.e. finding $x$ such that $g(x)=0$. To maximize $f$ (if continuous) we can solve $g(x) = f^\prime (x)=0$.} \\ \\

\underline{Bisection:}
\begin{itemize}
	\item For 1-dimensional continuous functions.
	\item Let $g:\;\mathbb{R} \rightarrow \mathbb{R}$ be a continuous function on $[a,b]$. We want to find $x_*$ such that $g(x_*)=0$.
	\item Idea: Find $l$ and $u$ such that $g(l)\cdot g(u) <0$, which implies that $g(l)$ and $g(u)$ will have different signs. 
	\item Set $c = \frac{l+u}{2}$, and compute $g(c)$. If $g(l)\cdot g(c) <0$, then set $u=c$. Otherwise, set $l=c$. 
	\item Repeat the step above. 
	\item Pros: Easy to code, and understand. Converges in linear time. We only need continuity, not differentiability. 
	\item Cons: Could be multiple roots. Only works in 1-dimension, doesn't generalize nicely to higher dimensions. Doesn't use much information about other values. For example, if $g(l)=-0.1$, and $g(u)=10000$, then we still select $c$ to be in the middle. 
\end{itemize}
\vspace{0.3in} 
\underline{Newton-Raphson}

\begin{itemize}
	\item This is an iterative algorithm to solve for $g(x)=0$.
	\item Idea: Update $x_t$ to $x_{t+1}$, where $x_{t+1} = x_t + \eta_t$. 
	\item How to choose $\eta_t$ is the question.
	\item We can write 
				$$g(x_{t+1}) = g(x_t + \eta_t) \approx g(x_t) + \eta_t g^\prime(x_t) + \mathcal{O}(\eta^2_t)$$
				and ignoring the higher-order terms, if we set $g(x) + \eta_tg^\prime(x_t) = 0$, then we have 
				$$\eta_t = -\frac{g(x_t)}{g^\prime(x_t)}$$
	\item Algorithm:
	\begin{itemize}
		\item Pick $x_0$. Set $t=0$.
		\item Update $x_{t+1} = x_t - \frac{g(x_t)}{g^\prime(x_t)}$.
		\item If $|g(x_{t+1})| < \epsilon$, then stop. Otherwise, set $t \to t+1$ and update again. 
	\end{itemize}
	
	\item Pros: Typically fast (quadratic convergence). Works in multiple dimensions. Only needs one (or two) derivatives. 
	\item Cons: Sensitive to the choice of $x_0$. There could be multiple roots. We need to be able to calculate derivatives. 
	\item If $g:\mathbb{R}^m \rightarrow \mathbb{R}^m$, then $\vec{x}_{t+1} = \vec{x}_t - \left [\nabla g(\vec{x}_t) \right ] ^{-1} g(\vec{x}_t)$ 
	\item To maximize $l(\theta)$, we want to solve $l'(\theta) =0$, where $\theta_{t+1} = \theta_t - \left [l''(\theta_t) \right ] ^{-1} l'(\theta_t)$. 
\end{itemize}
\vspace{0.3in}
\underline{Rate of Convergence of a Sequence:}
Let $x_1$, $x_2, \dots, $ be a sequence that converges to some value $x_*$. Then we say that the sequence converges with quadratic rate if 
$$\underset{t\to \infty}{\lim} \frac{|x_{t+1} - x_* |}{|x_t - x_*|^2} = c, \; \; 0 < c < \infty$$ 
Similarly, we say that a sequence converges with a linear rate if 
$$\underset{t\to \infty}{\lim} \frac{|x_{t+1} - x_* |}{|x_t - x_*|} = c, \; \; 0 < c < 1$$
where if $c=1$ we say the sequence has a `super linear' rate of convergence. \\
Question: Are there algorithms that converge in cubic time?\\
Answer: Yes, but only for specific types of problems.  

\vspace{0.3in}

\underline{Scoring}

\begin{itemize}
	\item This is a small modification of the Newton-Raphson method, specifically for maximizing likelihoods. 
	\item In Newton-Raphson we had $\theta_{t+1} = \theta_t - \left [l''(\theta_t) \right ] ^{-1} l'(\theta_t)$, where in Scoring we use $\theta_{t+1} = \theta_t - I(\theta_t)^{-1} l'(\theta_t)$.
	\item $l''(\theta_t)$ is the \emph{observed} Fisher information, while $I(\theta_t)^{-1} = E(-l''(\theta))$ is the \emph{expected} Fisher information.
	\item Scoring is preferred to Newton-Raphson if the expected information is easier to compute than the observed (e.g. in exponential families).
	\item Scoring coverges linearly. 
\end{itemize}

\vspace{0.3in}

\underline{The EM Algorithm:}

\begin{itemize}
	\item For many problems, the likelihood itself can be difficult to compute. e.g.
	$$\eta_{ij} = x_{ij}^T\beta + z_{ij}^T \gamma_i $$
	$$y_{ij}|\beta, \gamma_j \sim Bin(n_{ij}, g^{-1}(\eta_{ij})) $$
	$$ \gamma_i \overset{i.i.d}{\sim} \mathcal{N}(0,\Sigma^{-1})$$
	where $\{y_i\}$ is the data, with parameters $\{\beta, \Sigma\}$, and latent variables $\{\gamma_i\}$. \\
	The likelihood for this model is then 
	\begin{align*}
	p(\vec{y}|\beta, \Sigma) 
	& = \int p(\vec{y},\{\gamma\}|\beta,\Sigma) \; d\gamma \\ 
	& = \int \prod_{i,j} {n_{ij} \choose y_{ij}} \left[g^{-1}(\eta_{ij}) \right] ^{y_{ij}} \left[1-g^{-1}(\eta_{ij}) \right] ^{n_{ij}-y_{ij}} \cdot \prod_j (2\pi)^{-p/2} |\Sigma|^{-1/2} \exp \left\{-\frac{1}{2}\gamma_j^T\Sigma^T\gamma_j \right \} \; \; d\gamma_1, \dots, d\gamma_j \\ 
	& = \text{nothing nice at all}
	\end{align*}
	\item Overall, our likelihood involves integrals that are difficult to compute. 
	\item For these situations it's hard to use Bisection or Newton-Raphson. Using EM we avoid directly computing the integrals. 
	\item Suppose we have a model with parameter $\theta$, observed data $y_{obs}$, and ``missing'' data $y_{mis}$ to maximize. 
				$$p(y_{obs} | \theta) = \int p(y_{obs},y_{mis} |\theta)\; dy_{mis}$$
				here, we can use the EM algorithm. 

            \item Define $Q(\theta|\theta^{(t)}) = E\left[ \log p(y_{obs},y_{mis}|\theta) | y_{obs},\theta^{(t)}\right] = \int \log p(y_{obs},y_{mis}|\theta) p(y_{mis}| y_{obs},\theta^{(t)}) dy_{mis}$. 
	
	Algorithm:
	
		\begin{itemize}
			\item Select $\theta ^{(0)}$, set $t=0$. 
			\item Set $\theta ^{(t+1)} = \underset{\theta}{argmax}\;\; Q(\theta|\theta^{(t)})$. 
			\item Check convergence. If $\frac{|\theta^{(t+1)}| - |\theta^{(t)}|}{|\theta^{(t)}|} < \epsilon$, then stop. Otherwise, increment $t \to t+1$ and go back to the previous step. 
		\end{itemize}
		
	\item Simple Example:
				$$y_{obs} | y_{mis} \sim \mathcal{N}(y_{mis},1)$$
				$$y_{mis} \sim \mathcal{N}(\theta, V), \quad \text{$V$ known.} $$
				\underline{Goal:} maximize $p(y_{obs}|\theta)$. 
				$$p(y_{obs}|\theta) = \int p(y_{obs},y_{mis}|\theta)\; dy_{mis}  = \int p(y_{obs}|y_{mis})p(y_{mis}|\theta)\; dy_{mis}.$$
	Here we have  
				\begin{align*}
				Q(\theta|\theta^{(t)}) 
				& = E\left[p(y_{obs}|y_{mis})p(y_{mis}|\theta) \log \right] \\
				& = E \left[-\frac{1}{2}(y_{obs}-y_{mis})^2 - \frac{1}{2} \log(2\pi) - \frac{1}{2V} \log(V) -\frac{1}{2}\log (2\pi) - \frac{1}{2V}(y_{mis} - \theta)^2 \bigg|y_{obs},\theta ^{(t)}\right ] \\
				& = E \left[-\frac{1}{2V}(y_{mis} - \theta)^2 \bigg|y_{obs},\theta ^{(t)}\right ] \\
				\end{align*}
where we have ignored any term not involving $\theta$. \\
To compute this expectation, we need to know $p(y_{mis}|y_{obs},\theta^{(t)})$. \\ 
$$p(y_{mis}|y_{obs},\theta^{(t)}) \propto p(y_{mis},y_{obs}|\theta^{(t)})$$
$$\implies y_{mis}|y_{obs},\theta^{(t)} \sim \mathcal{N} \left(\frac{\frac{\theta^{(t)}}{V} + y_{obs}}{\frac{1}{V}+1},\frac{1}{\frac{1}{V}+1} \right) \sim \mathcal{N} \left(\frac{\theta^{(t)}+Vy_{obs}}{V+1},\frac{V}{V+1} \right) $$
Thus, 
				\begin{align*}
				Q(\theta|\theta^{(t)}) 
				& = E \left[-\frac{1}{2V}(y_{mis} - \theta)^2 \bigg|y_{obs},\theta ^{(t)}\right ] \\
				& = -\frac{1}{2V}E \left[y_{mis}^2 + \theta^2 -2y_{mis}\theta \bigg|y_{obs},\theta ^{(t)}\right ] \\
				& = -\frac{1}{2V}\left(\theta^2 -2\theta E[y_{mis} | y_{obs},\theta^{(t)} ]  \right ) \\
				& = -\frac{1}{2V}\left(\theta^2 -2\theta \frac{\theta^{(t)}+Vy_{obs}}{V+1} \right ) + constant \\
				\end{align*}
So, 
$$\frac{dQ}{d\theta} = -\frac{1}{2V}\left(2\theta - 2\frac{\theta^{(t)}+Vy_{obs}}{V+1} \right )$$
and setting equal to $0$, we arrive at 
$$ \theta = \frac{\theta^{(t)}+Vy_{obs}}{V+1}$$. 

\underline{Algorithm:} 
$$\theta^{(t+1)} = \frac{1}{V+1} \theta^{(t)} + \frac{V}{V+1}y_{obs}$$
$$y_{obs} | y_{mis} \sim \mathcal{N}(y_{mis},1)$$
$$y_{mis} \sim \mathcal{N}(\theta, V) $$
and as $t \to \infty$, $\theta^{(t+1)} \to y_{obs}$. \\
This is a linear rate of convergence: $\frac{1}{V+1}$, with speed inversely proportional to the size of $V$. Note that low rates indicate fast convergence, rates close to 1 indicate slow convergence.

\end{itemize}


\end{document}





