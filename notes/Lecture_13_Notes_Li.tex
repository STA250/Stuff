\documentclass[12pt]{article}
\usepackage{setspace, amsmath, mathdots, amssymb, graphicx, multirow, gensymb, listings}

\begin{document}
\title{STA250 Lecture-13 Notes}
\author{Shuo Li}
\maketitle

\underline{Recap:}
\begin{itemize}
\item We saw that EM can be used to maximize certain forms of complicated likelihood. 
\item EM:  \\
$$\theta^{(t+1)}=\underset{\theta}{argmax}\; Q(\theta|\theta^{(t)})$$
where $$Q(\theta|\theta^{(t)})=E[log\;P(Y_{obs}, Y_{mis}|Y_{obs}, \theta^{(t)})]$$
$$=\int log\;P(Y_{obs}, Y_{mis}|\theta)*P(Y_{mis}|Y_{obs}, \theta^{(t)})dY_{mis}$$
Note: EM maximize $log\;P(Y_{obs}|\theta)$ by expanded log-likelihood $log\;P(Y_{obs}, Y_{mis}|\theta)$\\
where the observed data likelihood preserved\\
i.e. $$\int P(Y_{obs}, Y_{mis}|\theta)dY_{mis}=P(Y_{obs}|\theta)$$
\end{itemize}

\underline {Two key points:}
\begin{itemize}
\item $Y_{mis}$ dos not have to correspond to "real" missing data
\item The choice of $Y_{mis}$ is not unique
\end{itemize}

\underline {Example:}\\
\begin{itemize}
\item Model-1: \\
$Y_{obs}|\theta \sim N(\theta, v+1)$\\
Goal: find MLE for $\theta$ (answer is $Y_{obs}$)\\
No missing data!\\
Consider a "complete" model s.t.
$$Y_{obs}|Y_{mis}\sim N(Y_{mis}, 1)$$
$$ Y_{mis} \sim N(\theta, v)$$
Need to check:
$$\int P(Y_{obs}, Y_{mis}|\theta)dY_{mis}=P(Y_{obs}|\theta)$$
We can show (standard result) that this is true here.\\
Here $Y_{mis}$ is not "real" missing data.\\
\end{itemize}
What is we instead used a different "complete" data model?
\begin{itemize}
\item Model-2:\\
$$Y_{obs}| \widetilde{Y}_{mis}, \theta \sim N( \widetilde{Y}_{mis}+ \theta, v)$$
$$\widetilde{Y}_{mis}\sim N(0, 1)$$
We can show that again:
$$\int P(Y_{obs}, \widetilde{Y}_{mis}|\theta)d\widetilde{Y}_{mis}=P(Y_{obs}|\theta)$$
For this "complete" data model, the EM algorithm is:
$$\theta^{(t+1)}=(\dfrac{v}{v+1})\theta^{(t)}+(\dfrac{1}{v+1})Y_{obs}$$
\end{itemize}
We have tow EMs corresponding to two "complete" data models. Both give same MLE, which is better?
\begin{itemize}
\item M-1 has linear convergence rate $\dfrac{1}{v+1}$ 
\item M-2 has linear convergence rate $\dfrac{v}{v+1}$ 
\end{itemize}
Lower is better, depend on $v$. 
\begin{itemize}
\item M-1 is know as a sufficient augmentation scheme ($Y_{mis}$ is a sufficient statistic for $\theta$ in the "complete" data model)
\item M-2 is know as an ancillary augmentation scheme (Since $\widetilde{Y}_{mis}$ does not depend on $\theta$)
\end{itemize}
It turns out that the EM algorithm has an important property: Monotone convergence.\\
i.e. $$l(\theta^{(t+1)})\geqslant l(\theta^{(t)})$$
where $$l(\theta)=log\;P(Y_{obs}|\theta)$$
This makes EM very stable (\& popular); N-R, Bisection, Scoring etc. do not have this property. \\

\underline{Proof:}\\\\
Note: $$P(Y_{obs}, Y_{mis}|\theta)=P(Y_{obs}|\theta)P(Y_{mis}|Y_{obs}, \theta)$$
$$\Rightarrow l_{obs}(\theta)= log\;P(Y_{obs}, Y_{mis}|\theta)-log\;P(Y_{mis}| Y_{obs},\theta)$$
Integrate both sides w.r.t. $P(Y_{mis}|Y_{obs}, \theta^{(t)})$\\
$$l_{obs}(\theta)= Q(\theta|\theta^{(t)})+H(\theta|\theta^{(t)})$$
where
$$H(\theta|\theta^{(t)})=-\int log\;P(Y_{mis}|Y_{obs},\theta)P(Y_{mis}|Y_{obs},\theta^{(t)})dY_{mis}$$
So,
$$l_{obs}(\theta^{(t+1)})-l_{obs}(\theta^{(t)})=[Q(\theta^{(t+1)}|\theta^{(t)})-Q(\theta^{(t)}|\theta^{(t)})]+[H(\theta^{(t+1)}|\theta^{(t)})-H(\theta^{(t)}|\theta^{(t)})]$$
First term $\Delta Q$ is $\geqslant 0$ by definition of Q function.
We only need to show $\Delta H = H(\theta^{(t+1)}|\theta^{(t)})-H(\theta^{(t)}|\theta^{(t)}) \geqslant 0 $
$$\Delta H =\int log\;(\dfrac{P(Y_{mis}|Y_{obs}, \theta^{(t)})}{P(Y_{mis}|Y_{obs}, \theta^{(t+1)})})P(Y_{mis}|Y_{obs}, \theta^{(t)})dY_{mis}$$
This is the KL divergence $KL(P(Y_{mis}|Y_{obs}, \theta^{(t)}))||P(Y_{mis}|Y_{obs}, \theta^{(t+1)})$\\
$\Rightarrow$ By properties of KL divergence $ \Delta H \geqslant 0$ with $\Delta H=0$ \\
iff. $$P(Y_{mis}|Y_{obs}, \theta^{(t+1)})=P(Y_{mis}|Y_{obs}, \theta^{(t)})$$
Therefore, $$l_{obs}(\theta^{(t+1)})-l_{obs}(\theta^{(t)})\geqslant 0$$\\

\underline {Aside:} \\
We can also use EM to find posterior modes not just MLE's.
\begin{itemize}
\item To maximize $log\;P(\theta|Y_{obs})$,\\
Let $$Q_{MAP}(\theta|\theta^{(t)})=E[log\;P(\theta, Y_{mis}|Y_{obs})|Y_{obs}, \theta^{(t)}]$$
$$=\int log\;P(\theta, Y_{mis}|Y_{obs})P(Y_{mis}|Y_{obs},\theta^{(t)})dY_{mis}$$
\item "MAP estimate" maximize a posterior value (i.e. posterior mode)
\end{itemize}

\underline {Example:} 
\begin{itemize}
\item Probit Regression
$$Y_i|X_i \sim Bin(1, g(X_i^T\beta))$$ \\
For logistic regression: $g(u)=\dfrac{e^u}{1+e^u}$\\
For probit regression:  $g(u)=\Phi(u)$, CDF of $N(0, 1)$\\
Form a complete data model:\\
$$Y_i|Z_i, \beta \sim 1_{\{z_i\geqslant 0\}}$$
$$Z_i| \beta \sim N(X_i^T\beta, 1)$$
Parameter: $\beta$\\
Complete data: $\{(Y_i, Z_i), i=1, 2, ..., n\}$\\
Observed data: $\{(Y_i), i=1, 2, ..., n\}$\\
Missing data: $\{(Z_i), i=1, 2, ..., n\}$\\
\item Check:
$$\int P(Y_i, Z_i|\beta)dZ_i=P(Y_i|\beta)$$
$$P(Y_i=1|\beta)=\int_{Z>0}\dfrac{1}{\sqrt{2\pi}}exp(-\frac{1}{2}(Z-X_i^T\beta)^2)dZ_i=\Phi(X_i^T\beta)$$
$\Rightarrow$ preserves observed data log-likelihood\\\\
Let's derive the EM algorithm for this model:\\ 
$$Q(\theta|\theta^{(t)})=E[log\;P(Y_{obs}, Y_{mis}|\theta)|Y_{obs}, \theta^{(t)}]$$
$$Q(\beta|\beta^{(t)})=E[log\;P(Y, Z|\beta)|Y, \beta^{(t)}]$$
Take the expectations, we need to know $Z_i|Y_i,\beta^{(t)}$\\
$$Z_i|Y_i=0,\beta^{(t)}\sim TN(X_i^T\beta^{(t)}, 1, (-\infty, 0])$$
$$Z_i|Y_i=1,\beta^{(t)}\sim TN(X_i^T\beta^{(t)}, 1, [0, +\infty))$$
$$Q(\beta|\beta^{(t)})=-E[\frac{1}{2}(Z_i-X_i^T\beta)^2|Y, \beta^{(t)}]$$
$\Rightarrow $Maximizer of $Q(\beta|\beta^{(t)})$\\
We can show,\\
If $Y_i=1$
$$Z_i^{(t+1)}=X_i^T\beta^{(t)}+\dfrac{\Phi(X_i^T\beta^{(t)})}{1-\Phi(-X_i^T\beta^{(t)})}$$
If $Y_i=0$
$$Z_i^{(t+1)}=X_i^T\beta^{(t)}+\dfrac{\Phi(X_i^T\beta^{(t)})}{\Phi(-X_i^T\beta^{(t)})}$$
The maximizer of $Q(\beta|\beta^{(t)})$ $w.r.t.\; \beta$ is seen to be the LSE of $\beta$ when regressing $Z^{(t+1)}$ on X.\\
i.e.
$$\beta^{(t+1)}=(X^TX)^{-1}X^TZ^{(t+1)}$$
where
$$Z^{(t+1)}=[\begin{array} {cccc} {Z_1}^{(t+1)}\\\vdots\\{Z_n}^{(t+1)} \end{array}]$$
E-Step: Compute $Z^{(t+1)}$\\
M-Step: Compute $\beta^{(t+1)}=(X^TX)^{-1}X^TZ^{(t+1)}$\\
\end{itemize}
\end{document}
