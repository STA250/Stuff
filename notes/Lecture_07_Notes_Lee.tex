\documentclass[11pt,a4paper]{article}

\usepackage[margin=3cm]{geometry}
\usepackage{setspace}
\usepackage{amssymb,amsmath, amsthm}
\usepackage{graphicx,picinpar,epsf}
\usepackage{float}
\usepackage{fancybox}


\begin{document}

\title{\textbf{STA 250 Lecture Notes(Oct. 21)}}
\author{Olivia Lee}
\date{}
\maketitle
\noindent
\textbf{\Large Metropolis-Within-Gibbs}
\begin{itemize}
\item Allows you to sample from a high-dimensional distribution using a sequence of lower dimensional distributions.
\item Generalizes Gibbs sampler in two ways\begin{enumerate}
                                             \item If we can't sample directly/exactly, then we can use MH.
                                             \item Sample sub-blocks of parameters, not necessary full conditionals. \\
                                             (ie)\,$P(\theta_1,\theta_2,\phi | \overrightarrow{y})$\\
                                             $\rightarrow \text{sample}  P(\theta_1,\theta_2 | \phi, \overrightarrow{y})$ (exactly)\\
                                             $\rightarrow \text{sample}  P(\phi | \theta_1,\theta_2  \overrightarrow{y})$ (using MH)\\
                                             \end{enumerate}
\end{itemize}
\textbf{\underline{Idea of Homework:}}\\
$\,$Two Strategies\begin{enumerate}
  \item  Sample from $P(\overrightarrow{\beta}|\overrightarrow{y})$ using MH and a multivariate proposal for $\overrightarrow{\beta}$
  \item For $j=1,...,p$ sample from $P(\beta_j|\beta_{[-j]},\overrightarrow{y})$
\end{enumerate}
$\star$ Recommend to do both in homework!\\
\\
\textbf{Q:}When to use non-symmetric proposals?\\
\textbf{A:}When $\theta$ has compact support then proposal can respect the boundaries.\\
\indent Ex: $X_i\sim Posi(\lambda) i=1,...,n.$ $\,$ Prior $\lambda \sim t_\upsilon I_{\{(0,\infty)\}}$\\
\indent $\,\quad\,\,$ Use MH, proposal $\theta^{\star}\sim TN(\theta^{(t)},v^2,[0,\infty))$ $\Rightarrow$ No longer symmetric!\\
\\
\textbf{\underline{Checking your MCMC code}}\\
\noindent For one dataset, how to know we reach convergence?\begin{enumerate}
  \item By eye, using traceplots.
  \item Use effective sample size to gauge roughly "how well" converged
  \item Other diagnostics : run multiple Markov Chains then compare if they are the same(Gelmen-Rubin)
\end{enumerate}
$\star$ If pass these tests, it doesn't necessary mean convergence. But if fail to pass, no converge.\\
\\
\noindent
Fortunately, we can use \textbf{simulation studies} to check if everything is working!\\
\\
\underline{\textbf{Idea:}}\begin{enumerate}
                   \item Simulation $\theta_{(j)}$ from the prior $P(\theta)$
                   \item Simulate a dataset $\overrightarrow{y}_{(j)}$ from the model $P(y|\theta_{(j)})$
                   \item Sample from posterior for dataset $\overrightarrow{y}_{(j)}$
                   \item Find $100(1-\alpha)\%$ central credible interval for $\theta$
                   \item Record yes or no that the interval contained $\theta_{(j)}$
                   \item Check roughly $100(1-\alpha)\%$ of intervals contained their specific $\theta_{(j)}$
                 \end{enumerate}
\underline{Why does this work?}
\begin{align*}
    &\int P(\theta) \int I_{\{\theta\in S^{1-\alpha}(\overrightarrow{y})\}}P(\overrightarrow{y}|\theta)dyd\theta\\
    =& \int \int I_{\{\theta\in S^{1-\alpha}(\overrightarrow{y})\}}P(\theta)P(\overrightarrow{y}|\theta)dyd\theta\\
    =&\int P(y)[\int I_{\{\theta\in S^{1-\alpha}(\overrightarrow{y})\}}P(\overrightarrow{y}|\theta)d\theta]dy\\
    =&(1-\alpha) \int P(y)dy \\
    =& 1-\alpha\\
\end{align*}
\textbf{Comparison}\\
\underline{Bayes:} $\qquad \int P(\theta) \int I_{\{\theta\in S^{1-\alpha}(\overrightarrow{y})\}}P(\overrightarrow{y}|\theta)dyd\theta = 1-\alpha$\\
\underline{Frequentist:} $\quad \int I_{\{\theta\in C^{1-\alpha}(\overrightarrow{y})\}}P(y|\theta)dy = 1-\alpha \quad \forall\theta$ \\
\begin{figure}[H]
  \centering
  \includegraphics[width=4in]{CI.jpeg}
  \caption{$95\%$ Credoble Interval vs $95\%$ Confidence Interval}
 \end{figure}
\noindent
\textbf{\underline{Posterior Predictive Checking}}\\
$\bullet$ Validation simulation checks you can sample from the posterior under your model. But it doesn't tell you if your model is a good fit to the data.\\
\\
\noindent
\underline{Idea:}\\
Having fit your model to the data, you know roughly what $\theta$ is, so if you simulate from $P(y|\theta)$ the simulated data should look "similar" to the real data.\\

\underline{Formally:} $P(\widetilde{y}|\overrightarrow{y})=\int P(\widetilde{y},\theta|\overrightarrow{y}) d\theta =\int P(\widetilde{y}|\theta)P(\theta|\overrightarrow{y})d\theta$\\
\\
\underline{Recipe:}\\
$\bullet$Sample $\theta^{(t)}$ from the posterior $P(\theta|\overrightarrow{y})$\\
$\bullet$For each$\theta^{(t)}$, sample a new dataset from $P(\overrightarrow{y}|\theta^{(t)})$\\
$\bullet$We now have is m predictive datasets and 1 real dataset.\\
$\bullet$We can take univariate summary statistics of each dataset and compare to the real dataset\\
\begin{figure}[H]
  \centering
  \includegraphics[width=4in]{hist.jpeg}
 \end{figure}
\indent $\star$ Could use mean, median, min, max.\\
\indent $\star$ Compare the posterior predictive p-value.\\
\underline{\textbf{Posterior Predictive P-value:}} \\
\begin{equation*}
    p = 2 * min\{P(T(y^*)>T(\overrightarrow{y})),P(T(y^*)<T(\overrightarrow{y}))\}
\end{equation*}
Heuristically, if p is small then your model is not a good fit of the data.
\end{document}
This is never printed 