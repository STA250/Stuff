\documentclass[11pt, oneside]{article}   	% use "amsart" instead of "article" for AMSLaTeX format
\usepackage{geometry}                		% See geometry.pdf to learn the layout options. There are lots.
\geometry{letterpaper}                   		% ... or a4paper or a5paper or ... 
%\geometry{landscape}                		% Activate for for rotated page geometry
%\usepackage[parfill]{parskip}    		% Activate to begin paragraphs with an empty line rather than an indent
\usepackage{graphicx}	
\usepackage{listings}			% Use pdf, png, jpg, or eps§ with pdflatex; use eps in DVI mode
								% TeX will automatically convert eps --> pdf in pdflatex		
\usepackage{amssymb}

\title{STA 250 Lecture 18 Notes}
\author{Chuan Qin}
\date{December 2, 2013}							% Activate to display a given date or no date

\begin{document}
\maketitle
%\section{}
%\subsection{}
\begin{enumerate}
\item \textbf{Notes on Homework 4}:
\begin{itemize}
\item Only need to write one kernel which obtains samples from a truncated normal distribution for both Q1 and Q2.
\item Your code for the kernel in Q1 has to be robust/error free or else it might cause problem when used for Q2.
\item \underline{Probit MCMC}:
\[
Y_i | Z_i = 1_{\{Z_i>0\}}
\]
\[
Z_i | \beta \sim N(x_i^T\beta,1)
\]
\[
\beta_i \sim N(\beta_0,\Sigma_0)
\]
\[
\Rightarrow P(\beta | Z,Y) \sim \mathrm{Normal}
\]
\[
\Rightarrow P(Z_i | \beta_i,Y_i) \sim \mathrm{Truncated \ Normal}
\]
\underline{MCMC}:
\begin{center}
\begin{verbatim}
for (iter in (niter + burnin)){
    if (use GPU){
    z = rtruncnormGPU(...)   # CUDA and kernel in (...)
    } else{
    z = rtruncnormCPU(...)  # regular R/Python in (...)
    }
    beta = rmvnorm(...)
}
\end{verbatim}
\end{center}
\end{itemize}
\vspace{1cm}

\item \textbf{C/C++}
\begin{itemize}
\item C is a very fast complied language.
\item Data types need to be explicitly defined.
\item Vectors/matrices are typically implemented using ``pointers''.
\item Pointers point to memory locations, from which you can look up values at those locations.
\item About homework: \begin{itemize}
\item ``\underline{\hspace{0.4cm}}global\underline{\hspace{0.4cm}} void'' tells the compiler that this function is a kernel and it does not return any value.
\item The samples are written into the memory locations pointed by the input arguments, defined as pointers.
\end{itemize}
\end{itemize}
\vspace{1cm}

\item \textbf{Truncated Normal Sampling}\vspace{0.2cm} \\
If 
\[
x\sim N(\mu, \sigma^2) 1_{\{x\in (a,b)\}},
\]
then
\[
x \sim \mathrm{Truncated-Normal}(\mu,\sigma^2;(a,b)).
\] 
The simplest sampling method to implement is rejection sampling, which repeatedly sample from $N(\mu, \sigma^2)$ until the value falls in the interval $(a,b)$:
\begin{verbatim}
accepted = False
numtries = 0
# Specify maxtimes as the maximum number of attempts
while (! accepted and numtries < maxtimes ){    
    numtimes = numtimes + 1
    x = rnorm(mu, sigma)
    if (x>=a and x<=b){
    accepted = True
    } 
}
\end{verbatim}
However, this method is quite inefficient when $a$ is several standard deviations larger than $\mu$ or $b$ is several standard deviations smaller than $\mu$. Instead, it is more advisable to use the following \emph{rejection sampling} algorithm for sampling from tail truncated normal distribution.

\underline{Rejection Sampling} \vspace{0.2cm}\\
To sample from a distribution with p.d.f. $f(x)$, if we can find another distribution with p.d.f. $g(x)$ such that 
\[
f(x)\leq Mg(x), \hspace{1cm} \forall x,
\]
then we can use $g$ to sample from $f$ as follows:
\begin{itemize}
\item[(i)] Sample a value $x^*$ from $g(x)$.
\item[(ii)] Sample $U\sim U[0,1]$.
\item[(iii)] If 
\[
U \leq \frac{f(x^*)}{Mg(x^*)},
\]
then accept $x^*$. Otherwise return to (i). 
\end{itemize}
Remark: \vspace{0.1cm}\\
It is clear that the bigger the gap between $f(x)$ and $Mg(x)$, the lower the accepting probability. Therefore, ideally, we should choose $g(x)$ so that $f(x)$ and $Mg(x)$ are ``close'' for all $x$.  \vspace{0.1cm}\\

Robert (2009) proposed the following algorithm for sampling from a one-sided truncated normal distribution.

\underline{Rejection Sampling One-sided Truncated Normal} \vspace{0.2cm}\\
To sample from $X\sim N(0,1;(\mu^-,\infty))$,
\begin{itemize}
\item[(i)] Generate
\[
z = \mu^- + \mathrm{Expo}(\alpha).
\]
\item[(ii)] Compute
\[
\Psi(z) = \left\{\begin{array}{cl}
\displaystyle \exp\left(-\frac{(\alpha-z)^2}{2}\right), & \mbox{if   } \mu^- < \alpha  \vspace{0.2cm}\\
\displaystyle \exp\left(-\frac{(\alpha-z)^2}{2}-\frac{(\mu^- - \alpha)^2}{2}\right), & \mbox{if   } \mu^- \geq \alpha
\end{array}\right.
\]
\item[(iii)] If $U[0,1] < \Psi(z)$, accept; else try again. 
\end{itemize}
Optimal choice of $\alpha$: 
\[
\alpha^* = \frac{\mu^- + \sqrt{(\mu^-)^2+4}}{2}.
\]
Remark: \vspace{0.1cm}\\
In homework we need to sample from $N(\mu,\sigma^2;(a,\infty))$. To do this we can sample a value $x$ from
\[
N\left(0,1;\left(\frac{a-\mu}{\sigma},\infty\right)\right)
\]
and then use $\mu+\sigma x$ as our sample.

\end{enumerate}

\end{document}  